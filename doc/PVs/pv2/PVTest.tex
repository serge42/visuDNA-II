%%%%%%%%%%%%%%%%%%%%%%%%%%%%%%%%%%%%%%%%%
% Structured General Purpose Assignment
% LaTeX Template
%
% This template has been downloaded from:
% http://www.latextemplates.com
%
% Original author:
% Ted Pavlic (http://www.tedpavlic.com)
%
% Note:
% The \lipsum[#] commands throughout this template generate dummy text
% to fill the template out. These commands should all be removed when
% writing assignment content.
%
%%%%%%%%%%%%%%%%%%%%%%%%%%%%%%%%%%%%%%%%%

%----------------------------------------------------------------------------------------
%	PACKAGES AND OTHER DOCUMENT CONFIGURATIONS
%----------------------------------------------------------------------------------------

\documentclass{article}

\usepackage{fancyhdr} % Required for custom headers
\usepackage{lastpage} % Required to determine the last page for the footer
\usepackage{extramarks} % Required for headers and footers
\usepackage{graphicx} % Required to insert images
\usepackage{lipsum} % Used for inserting dummy 'Lorem ipsum' text into the template
%% Use this packages to read accented UTF-8 inputs
\usepackage[utf8]{inputenc}
\usepackage[T1]{fontenc}

% Margins
\topmargin=-0.45in
\evensidemargin=0in
\oddsidemargin=0in
\textwidth=6.5in
\textheight=9.0in
\headsep=0.25in

\linespread{1.1} % Line spacing

% Set up the header and footer
\pagestyle{fancy}
\lhead{\hmwkAuthorName} % Top left header
\chead{\hmwkTitle} % Top center header
\rhead{\date{01.10.2017}} % Top right header
\lfoot{\lastxmark} % Bottom left footer
\cfoot{} % Bottom center footer
\rfoot{Page\ \thepage\ of\ \pageref{LastPage}} % Bottom right footer
\renewcommand\headrulewidth{0.4pt} % Size of the header rule
\renewcommand\footrulewidth{0.4pt} % Size of the footer rule

\setlength\parindent{0pt} % Removes all indentation from paragraphs

%----------------------------------------------------------------------------------------
%	DOCUMENT STRUCTURE COMMANDS
%	Skip this unless you know what you're doing
%----------------------------------------------------------------------------------------

% Header and footer for when a page split occurs within a problem environment
\newcommand{\enterProblemHeader}[1]{
\nobreak\extramarks{#1}{#1 continued on next page\ldots}\nobreak
\nobreak\extramarks{#1 (continued)}{#1 continued on next page\ldots}\nobreak
}

% Header and footer for when a page split occurs between problem environments
\newcommand{\exitProblemHeader}[1]{
\nobreak\extramarks{#1 (continued)}{#1 continued on next page\ldots}\nobreak
\nobreak\extramarks{#1}{}\nobreak
}

\setcounter{secnumdepth}{0} % Removes default section numbers
\newcounter{homeworkProblemCounter} % Creates a counter to keep track of the number of problems

\newcommand{\homeworkProblemName}{}
\newenvironment{homeworkProblem}[1][Problem \arabic{homeworkProblemCounter}]{ % Makes a new environment called homeworkProblem which takes 1 argument (custom name) but the default is "Problem #"
\stepcounter{homeworkProblemCounter} % Increase counter for number of problems
\renewcommand{\homeworkProblemName}{#1} % Assign \homeworkProblemName the name of the problem
\section{\homeworkProblemName} % Make a section in the document with the custom problem count
\enterProblemHeader{\homeworkProblemName} % Header and footer within the environment
}{
\exitProblemHeader{\homeworkProblemName} % Header and footer after the environment
}

\newcommand{\problemAnswer}[1]{ % Defines the problem answer command with the content as the only argument
\noindent\framebox[\columnwidth][c]{\begin{minipage}{0.98\columnwidth}#1\end{minipage}} % Makes the box around the problem answer and puts the content inside
}

\newcommand{\homeworkSectionName}{}
\newenvironment{homeworkSection}[1]{ % New environment for sections within homework problems, takes 1 argument - the name of the section
\renewcommand{\homeworkSectionName}{#1} % Assign \homeworkSectionName to the name of the section from the environment argument
\subsection{\homeworkSectionName} % Make a subsection with the custom name of the subsection
\enterProblemHeader{\homeworkProblemName\ [\homeworkSectionName]} % Header and footer within the environment
}{
\enterProblemHeader{\homeworkProblemName} % Header and footer after the environment
}

%----------------------------------------------------------------------------------------
%	NAME AND CLASS SECTION
%----------------------------------------------------------------------------------------

\newcommand{\hmwkTitle}{PV séance du 04.11.2017} % title
\newcommand{\hmwkAuthorName}{S. Bouquet} % Your name
%% Puts on horizontal line
\newcommand{\hLine}{\noindent\makebox[\linewidth]{\rule{\textwidth}{.1pt}} \\}

%----------------------------------------------------------------------------------------
%	TITLE PAGE
%----------------------------------------------------------------------------------------

\title{
\vspace{2in}
\textmd{\textbf{\hmwkTitle}}\\
\vspace{3in}
}

\author{\textbf{\hmwkAuthorName}}
\date{\today} % Insert date here if you want it to appear below your name

%----------------------------------------------------------------------------------------

\begin{document}

\maketitle

%----------------------------------------------------------------------------------------
%	TABLE OF CONTENTS
%----------------------------------------------------------------------------------------

%\setcounter{tocdepth}{1} % Uncomment this line if you don't want subsections listed in the ToC

% \newpage
% \tableofcontents
 \newpage

%----------------------------------------------------------------------------------------
%	PROBLEM 1
%----------------------------------------------------------------------------------------

% To have just one problem per page, simply put a \clearpage after each
\large\textbf{Participants:}\normalsize~~B. Wolf, J. Stoppani, F. Wagen, D. Atlan, \hmwkAuthorName\\
\large\textbf{Excusés:}\normalsize~~P. Kuonen\\
\hLine

\section{Points abordés}
\begin{enumerate}
  \item Discussion du cahier des charges.
  \item Discussion des buts du projets.
  \item Discussion sur les différents graphes utiles, pas seulement les interactomes.
  \item Idée de réaliser une application web.
\end{enumerate}

\section{Décisions}
\begin{enumerate}
  \item Besoin de réévaluer les outils disponibles; rechercher pourquoi Gephi avait été choisi pour le premier projet.
  \item Définir un UseCase de l'application.
  \item Ajout des participants sur les PVs.
  \item Ajout des dates et des personnes responsables pour les tâches à réaliser.
  \item Les graphes peuvent devenir relativement grands (plus de 100'000 noeuds); nécessité d'avoir des fonctions de filtre pour isoler une partie du graphe.
  \item L'application doit pourvoir réaliser des graphes augmentés en intégrant les données du patient.
\end{enumerate}

\section{Tâches}
Pour la prochaine séance:
\begin{enumerate}
  \item \textit{B. Wolf, J. Stoppani:} récupérer un exemple de données à traiter. (jusqu'au 11.10.2017)
  \item \textit{J. Stoppani:} ajouter M. Wagen à la forge du projet (terminée).
  \item \textit{\hmwkAuthorName:} Lire le rapport du projet VisuDNA et tester le fonctionnement des diagrammes qu'il crée (jusqu'au 11.10.2017).
  \item \textit{\hmwkAuthorName:} Recherche d'informations sur les différents outils qu'on pourrait utiliser pour réaliser le projet et définir lequel est le plus adapté. (jusqu'au 18.10.2017)
  \item \textit{\hmwkAuthorName:} Amélioration des buts est des tâches du cahier des charges. Les tâches doivent correspondre à celles du planning; ajout des périodes de vacance sur le planning. (jusqu'au 11.10.2017)
  \item \textit{\hmwkAuthorName:} Définir les cas d'utilisation (jusqu'au 14.10.17)
\end{enumerate}

\section{Points ouverts}
\begin{itemize}
  \item Faut-il récupérer le programme de pré-traitement du projet précédent si cela fait sens dans ce nouveau projet (à définir d'après les cas d'utilisation).
  \item Choix de l'outil à utiliser (Gephi, Gama, Cytoscape, travail de J. Stoppani ou autre)
  \item Choix entre une application web ou locale
\end{itemize}


%----------------------------------------------------------------------------------------
\begingroup
  \renewcommand{\section}[2]{}
  \bibliographystyle{model1-num-names}
  \bibliography{sample}
  \nocite{*}
\endgroup

\end{document}
