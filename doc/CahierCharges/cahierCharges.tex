%%%%%%%%%%%%%%%%%%%%%%%%%%%%%%%%%%%%%%%%%
% Structured General Purpose Assignment
% LaTeX Template
%
% This template has been downloaded from:
% http://www.latextemplates.com
%
% Original author:
% Ted Pavlic (http://www.tedpavlic.com)
%
% Note:
% The \lipsum[#] commands throughout this template generate dummy text
% to fill the template out. These commands should all be removed when
% writing assignment content.
%
%%%%%%%%%%%%%%%%%%%%%%%%%%%%%%%%%%%%%%%%%

%----------------------------------------------------------------------------------------
%	PACKAGES AND OTHER DOCUMENT CONFIGURATIONS
%----------------------------------------------------------------------------------------
% ! TEX root = ./cahierCharges.tex

\documentclass{article}

\usepackage{fancyhdr} % Required for custom headers
\usepackage{lastpage} % Required to determine the last page for the footer
\usepackage{extramarks} % Required for headers and footers
\usepackage{graphicx} % Required to insert images
\usepackage{lipsum} % Used for inserting dummy 'Lorem ipsum' text into the template
%% Packages used to read accented UTF-8 inputs
\usepackage[utf8]{inputenc}
\usepackage[T1]{fontenc}
% \usepackage{multicol} unused
% Packages for the Gantt diagram
\usepackage{pgfgantt}
\usepackage{xcolor}

% Margins
\topmargin=-0.45in
\evensidemargin=0in
\oddsidemargin=0in
\textwidth=6.5in
\textheight=9.0in
\headsep=0.25in

\linespread{1.1} % Line spacing

% Set up the header and footer
\pagestyle{fancy}
\lhead{\projAuthorName} % Top left header
\chead{\projName} % Top center header
\rhead{\date{01.10.2017}} % Top right header
\lfoot{\lastxmark} % Bottom left footer
\cfoot{} % Bottom center footer
\rfoot{Page\ \thepage\ of\ \pageref{LastPage}} % Bottom right footer
\renewcommand\headrulewidth{0.4pt} % Size of the header rule
\renewcommand\footrulewidth{0.4pt} % Size of the footer rule

\setlength\parindent{0pt} % Removes all indentation from paragraphs

%----------------------------------------------------------------------------------------
%	DOCUMENT STRUCTURE COMMANDS
%	Skip this unless you know what you're doing
%----------------------------------------------------------------------------------------

% Header and footer for when a page split occurs within a problem environment
\newcommand{\enterProblemHeader}[1]{
\nobreak\extramarks{#1}{#1 continued on next page\ldots}\nobreak
\nobreak\extramarks{#1 (continued)}{#1 continued on next page\ldots}\nobreak
}

% Header and footer for when a page split occurs between problem environments
\newcommand{\exitProblemHeader}[1]{
\nobreak\extramarks{#1 (continued)}{#1 continued on next page\ldots}\nobreak
\nobreak\extramarks{#1}{}\nobreak
}

\setcounter{secnumdepth}{0} % Removes default section numbers
\newcounter{homeworkProblemCounter} % Creates a counter to keep track of the number of problems

\newcommand{\homeworkProblemName}{}
\newenvironment{homeworkProblem}[1][Problem \arabic{homeworkProblemCounter}]{ % Makes a new environment called homeworkProblem which takes 1 argument (custom name) but the default is "Problem #"
\stepcounter{homeworkProblemCounter} % Increase counter for number of problems
\renewcommand{\homeworkProblemName}{#1} % Assign \homeworkProblemName the name of the problem
\section{\homeworkProblemName} % Make a section in the document with the custom problem count
\enterProblemHeader{\homeworkProblemName} % Header and footer within the environment
}{
\exitProblemHeader{\homeworkProblemName} % Header and footer after the environment
}

\newcommand{\problemAnswer}[1]{ % Defines the problem answer command with the content as the only argument
\noindent\framebox[\columnwidth][c]{\begin{minipage}{0.98\columnwidth}#1\end{minipage}} % Makes the box around the problem answer and puts the content inside
}

\newcommand{\homeworkSectionName}{}
\newenvironment{homeworkSection}[1]{ % New environment for sections within homework problems, takes 1 argument - the name of the section
\renewcommand{\homeworkSectionName}{#1} % Assign \homeworkSectionName to the name of the section from the environment argument
\subsection{\homeworkSectionName} % Make a subsection with the custom name of the subsection
\enterProblemHeader{\homeworkProblemName\ [\homeworkSectionName]} % Header and footer within the environment
}{
\enterProblemHeader{\homeworkProblemName} % Header and footer after the environment
}

%----------------------------------------------------------------------------------------
%	NAME AND CLASS SECTION
%----------------------------------------------------------------------------------------

\newcommand{\projName}{VisuDNA-II} % Assignment title
\newcommand{\projDate}{2017\textemdash2018}
\newcommand{\projAuthorName}{S. Bouquet} % Your name
\newcommand{\projSupervisors}{P. Kuonen, B. Wolf, J. Stoppani}
\newcommand{\projInitiator}{D. Atlan (PhenoSystem SA)}
\newcommand{\docTitle}{Cahier des charges}

%----------------------------------------------------------------------------------------
%	TITLE PAGE
%----------------------------------------------------------------------------------------

\title{
  \includegraphics[width=0.9\columnwidth]{Logo_HEIA}\\
  \vspace{1cm}
  \textmd{\textit{Filière informatique}} \\
  \vspace{2cm}
  \textmd{Projet de semestre d'automne}\\
  \textmd{\projDate}\\
  \vspace{1.5cm}
  \noindent\makebox[\linewidth]{\rule{\textwidth}{0.5pt}}\\
  \vspace{.5cm}
  \textmd{\textbf{\projName}}\\
  \textmd{\textbf{\docTitle}}\\
  \noindent\makebox[\linewidth]{\rule{\textwidth}{0.5pt}}\\
  % \normalsize\vspace{0.1in}\small{\hmwkDueDate}\\
  % \vspace{0.1in}\large{\textit{\hmwkClassInstructor\ \hmwkClassTime}}
  \Large
  \vspace{3cm}
  \textit{Responsables:} \projSupervisors \\
  \vspace{1cm}
  \textit{Externe:} \projInitiator \\
  \vspace{1cm}
  \textit{Autheur:} \projAuthorName \\
  \author{}
}
% \author{\textbf{\projAuthorName}}
\date{} % Insert date here if you want it to appear below your name

%----------------------------------------------------------------------------------------

\begin{document}

\maketitle
%----------------------------------------------------------------------------------------
%	TABLE OF CONTENTS
%----------------------------------------------------------------------------------------

%\setcounter{tocdepth}{1} % Uncomment this line if you don't want subsections listed in the ToC

% \newpage
% \tableofcontents
 \newpage

%----------------------------------------------------------------------------------------
%	PROBLEM 1
%----------------------------------------------------------------------------------------

% To have just one problem per page, simply put a \clearpage after each problem

\section{Context}
\subsection{Projet initial}
  Le projet initial, VisuDNA, avait pour but de mieux comprendre l'ADN et le génôme humain. Ceci peut être réaliser en analysant certaines séquences ADN commune à tous les êtres humains; c'est-à-dire l'interactome du génôme humain.  L'analyse des différents interactomes parmet notemment d'aider l'études des maladies génétiques par exemple en étant capable de prédire si un individu fait partie d'une catégorie de personnes à haut risque d'être atteint de quelque maladie génétique.
  \\
  La compréhension de l'interaction entre les gènes étant encore partielle, il est actuellement difficile de déterminer de manière conclusive les gènes responsables d'une certaine maladie. Ce projet avait donc pour but d'aider à la compréhension de ces interactions en permettant une représentation graphique de l'interactome.
\cite{Sisto:2014}

\subsection{Suite du projet}
  Ce premier projet a abouti à un résultat encourageant mais nécessitant encore certaines améliorations. Il faudra commencer par comprendre le travail réaliser précédemment puis rechercher s'il existe d'autres outils de visualisation de graphes plus performants pour développer le nouveau projet ou s'il y a une possibilité d'améliorer la solution précédente.

\section{Buts du projet}
  Ce projet ce base sur un projet de semestre précédent qui avait pour but la visualisation de graphes d'interactions avec Gephi. Ce projet permettait de visualiser les graphes mais était limité quant à la manipulation de ceux-ci et souffrait de problèmes de performances lors de la visualisation de grands graphes.\\
  Ce nouveau projet doit donc essayer de résoudre ces problèmes soit en utilisant un autre outil de visualisation de graphe que celui utilisé précédemment soit en améliorant la solution précédente. La possibilité de ne pas utiliser un outil existant est également envisageable si aucune des solutions disponibles ne correspond aux critères du projet bien que ce ne soit pas de prime abord l'alternative la plus désirée.
  \subsection{Objectifs}
  \begin{itemize}
    \item Compréhension du projet initial ainsi que des librairies et des outils utilisés.
    \item Création de prototypes permettant de choisir quel outil serait le plus adapté à la réalisation du projet.
    \item Conception du nouveau projet d'après les résultats des points précédents.
    \item Implémentaion du projet.
    \item Documentation du processus de développement.
  \end{itemize}

\section{Glossaire}
  \begin{itemize}
    \item \textit{Interactome}: contient les interactions entre les gènes.
    \item \textit{Interactome annoté}: contient l'interactome avec des annotations concernant les gènes, les interactions, les variants et les maladies associées.
    \item \textit{Graphe augmenté}: contient l'interactome (et possiblement d'autres graphes) annoté et augmenté des données d'un patient.
  \end{itemize}
\cite{Sisto:2014}

\section{Activités}
  \subsection{Spécifications}
    \begin{itemize}
      \item Tests de l'ancien projet (cf. annexe 1)
      \item Elaboration du cahier des charges.
    \end{itemize}
  \subsection{Analyse}
    \begin{itemize}
      \item Définition des cas d'utilisation de l'application.
      \item Création de prototypes pour les outils à départager pour la réalisation du projet.
    \end{itemize}
  \subsection{Conception}
    \begin{itemize}
      \item Définition du diagramme des classes d'après l'outil à utiliser durant ce projet.
      \item Définition des cas tests.
    \end{itemize}
  \subsection{Implémentation}
  Implémentation du projet.
  \subsection{Tests}
  Les tests à effectuer seront définits durant la conception du projet.
  \subsection{Documentation}
  Le rapport de l'application et les éventuels documents annexes seront réaliser
  tout au long du processus de développement.\\
  Un mode d'emploi sera écris à la fin de l'implémentation si nécessaire.
  \subsection{Préparation défense}

\section{Contraintes}
  Aucune contrainte spécifique n'a été spécifiée pour le moment.\\
  Ce projet, comme son prédécesseur, est actuellement prévu pour utiliser Java bien que rien n'empêche l'utilisation d'autres langages si cela est jugé nécessaire au cours du projet.

\newpage
\section{Planification}
\ganttset{
  group/.append style={orange},
  milestone/.append style={red},
  progress label node anchor/.append style={text=red}
}
\begin{ganttchart}[
  x unit=1mm,
  y unit title=5mm,
  y unit chart=6mm,
  hgrid, vgrid,
  title height=1,
  title label font=\bfseries\footnotesize,
  bar/.style={fill=blue!70},
  bar height=.7,
  progress label text={},
  group right shift=0,
  group top shift=0.7,
  group height=0.3,
  group peaks width={.7},
  milestone height=.7,
  milestone top shift=.2,
  milestone left shift=0,
  milestone right shift=1,
  time slot format=isodate,
  inline,
  calendar week text=A\currentweek,
  today label=Current Week,
  today rule/.style={draw=red, thick},
  today=2017-10-09,
  today label node/.append style={anchor=north west}
  ]{2017-09-26}{2018-02-09}
  \gantttitlecalendar{year, month=shortname} \\
  %top nodes
  \ganttbar[name=holiday-october-top, bar/.style={fill=none, draw=none}]{Octobre}{2017-10-23}{2017-10-29}
  \ganttbar[name=holiday-november-top, bar/.style={fill=none,draw=none}]{}{2017-11-01}{2017-11-01}
  \ganttbar[name=holiday-december1-top, bar/.style={fill=none,draw=none}]{}{2017-12-08}{2017-12-08}
  \ganttbar[name=holiday-december-top, bar/.style={fill=none, draw=none}]{Noel}{2017-12-25}{2018-01-07}

  %gantt chart content
  \ganttgroup[inline=false]{Spécifications}{2017-09-27}{2017-10-13} \\
  \ganttbar[progress=80, inline=false]{Cahier charges}{2017-09-27}{2017-10-10}\\
  \ganttbar[inline=false, progress=30]{Review ancien projet}{2017-10-08}{2017-10-13} \\
  \ganttmilestone[inline=false]{Rendu cahier}{2017-10-13} \\
  %ANALYSE
  \ganttgroup[inline=false]{Analyse}{2017-10-14}{2017-11-08} \\
  \ganttbar[progress=0, inline=false]{Review ancien projet}{2017-10-14}{2017-10-15} \\
  \ganttbar[progress=0,inline=false]{Cas d'utilisations}{2017-10-14}{2017-10-18}\\
  \ganttlinkedbar[progress=0, inline=false]{Prototypes}{2017-10-15}{2017-11-04}\\
  %1st PRESENTATION
  \ganttgroup[inline=false]{Présentation 1}{2017-11-05}{2017-11-08}\\
  \ganttbar[progress=0,inline=false]{Préparation}{2017-11-05}{2017-11-08} \\
  \ganttmilestone[inline=false,]{Presentation}{2017-11-08} \\
  %CONCEPTION
  \ganttgroup[inline=false]{Conception}{2017-11-09}{2017-12-25}\\
  \ganttbar[inline=false,progress=0]{Diag. Classes}{2017-11-09}{2017-11-16}\\

  %IMPLEMENTATION
  \ganttgroup[inline=false]{Implémentation}{2017-11-23}{2018-01-14} \\
  %TESTS
  \ganttgroup[inline=false]{Tests}{2017-12-11}{2018-02-01} \\
  \ganttbar[inline=false,progress=0]{Unitaires}{2017-12-11}{2018-01-18} \\
  \ganttbar[inline=false,progress=0]{fonctionnels}{2018-01-12}{2018-02-01} \\
  %DOCUMENTATION
  \ganttgroup[inline=false]{Documentation}{2017-10-16}{2018-02-01}\\
  \ganttbar[inline=false,progress=0]{Ecriture rapport}{2017-10-16}{2018-02-01}\\
  \ganttbar[inline=false,progress=0]{Mode d'emploi}{2018-1-8}{2018-2-1}\\
  \ganttmilestone[inline=false]{Rendu rapport}{2018-02-01} \\

  \ganttgroup[inline=false]{Défense}{2018-02-02}{2018-02-07}\\
  \ganttbar[inline=false,progress=0]{Préparation}{2018-02-02}{2018-02-07}\\
  \ganttmilestone[inline=false]{Défense}{2018-02-08} \\

  % bottom nodes
  \ganttbar[name=holiday-october-bottom, bar/.style={fill=none, draw=none}]{}{2017-10-23}{2017-10-29}
  \ganttbar[name=holiday-november-bottom, bar/.style={fill=none,draw=none}]{}{2017-11-01}{2017-11-01}
  \ganttbar[name=holiday-december1-bottom, bar/.style={fill=none,draw=none}]{}{2017-12-08}{2017-12-08}
  \ganttbar[name=holiday-december-bottom, bar/.style={fill=none, draw=none}]{}{2017-12-25}{2018-01-07}

  % shading
  \begin{scope}
    \draw[opacity=.7,line width=7mm](holiday-october-top) -- ($(holiday-october-bottom)+(0,-5pt)$);
    % \draw[opacity=.7,line width=3](holiday-november-top) -- ($(holiday-november-bottom)+(0,-5pt)$);
    % \draw[opacity=.7,line width=3](holiday-december1-top) -- ($(holiday-december1-bottom)+(0,-5pt)$);
    \draw[opacity=.7,line width=15mm](holiday-december-top) -- ($(holiday-december-bottom)+(0,-5pt)$);
  \end{scope}
\end{ganttchart} \\
% \begin{ganttchart}[%Specs]
%     % Setting group if any
%     \ganttgroup[inline=false]{Group 1}{1}{5}\\
%     \ganttbar[progress=10,inline=false]{Planning}{1}{4}\\
%     \ganttmilestone[inline=false]{Milestone 1}{9} \\
%
%     \ganttgroup[inline=false]{Group 2}{6}{12} \\
%     \ganttbar[progress=2,inline=false]{test1}{10}{19} \\
%     \ganttmilestone[inline=false]{Milestone 2}{17} \\
%     \ganttbar[progress=5,inline=false]{test2}{11}{20} \\
%     \ganttmilestone[inline=false]{Milestone 3}{22} \\
%
%     \ganttgroup[inline=false]{Group 3}{13}{24} \\
%     \ganttbar[progress=90,inline=false]{Task A}{13}{15} \\
%     \ganttbar[progress=50,inline=false, bar progress label node/.append style={below left= 10pt and 7pt}]{Task B}{13}{24} \\ \\
%     \ganttbar[progress=30,inline=false]{Task C}{15}{16}\\
%     \ganttbar[progress=70,inline=false]{Task D}{18}{20} \\
% \end{ganttchart}


%----------------------------------------------------------------------------------------
\newpage
\section{Annexe}
\begin{enumerate}
  \item Résumé des choix du projet précédent.
\end{enumerate}

\newpage
\begingroup
  % Modifies the title of the "references" section
  \renewcommand{\section}[2]{\Large\textbf{Références}\normalsize}
  \bibliographystyle{model1-num-names}
  \bibliography{sample}
  \nocite{*}
\endgroup

\end{document}
